\subsection{Purpose}
\hspace{\parindent}This document provides a detailed description, mainly of the  architecture and the UI, of the 'Polaris' mobile application.\\
'Polaris' application is a system used to enhance travelling experience through idea sharing among people, exploration of other users' travels, documentation and easy manipulation of trip destinations and ideas. The system itself is made so the user has every single information regarding his travel in one place, without having to remember or worry about any details.\\
'Polaris' is mainly an Android mobile application, with a possibility of being also expanded as a web application and/or iOS application in the future.  
\newpage

\subsection{Scope}
\hspace{\parindent}'Polaris' is an application that helps all travellers around the world to easily and efficiently manage their travels and thus enhance their travelling experience. The target audience for this application is everyone who uses a smartphone and has at least some experience in using mobile applications, as some of the patterns and application usages might not be intended for the novices in the field. Thus the target audience ranges from 15 to 50 years old, although there are no strict boundaries.\linebreak


The application is mainly intended to be used when a user is planning and making their own trip and has a liberty to organize their free time and places to visit.\linebreak


Another important usage of the application is sharing trip ideas with friends and other users around the world. Planning trips by itself is a daunting and time consuming task, and with the lack of adequate applications on the market, we wanted to create something that is going to allow users to easily share and modify their previous trips and therefore improve the overall travelling experience for others. 
\break
The most important functions of the application are arranging a trip, finding points of interest in the area, organizing visits to those points, exploring accommodation and restaurants in the area, and crafting your own views of the travel by providing additional comments on the whole experience.


\newpage

\subsection{Definitions, Acronyms, Abbreviations}
\subsubsection{Definitions}
\begin{itemize} 
	\item \textbf{Application}: a computer (mobile) program that is designed for a particular purpose. 
	\item \textbf{Smartphone}: a mobile phone that performs many of the functions of a computer, typically having a touchscreen interface, internet access, and an operating system capable of running downloaded apps. 
	\item \textbf{Google Maps}: a web mapping service developed by Google, used both as a standalone app and as an integrated mapping solution in most of the apps.
	\item \textbf{iOS}: operating system developed by Apple, used by their portable devices like iPads and iPhones.
	\item \textbf{Android}: most popular operating system for smartphones and tablets, developed by Google and partners.
	\item \textbf{Backend}: the part of a computer system or application that is not directly accessed by the user, typically responsible for storing and manipulating data.
\end{itemize}
\subsubsection{Acronyms}
\begin{itemize}
	\item \textbf{API}: Application programming interface, computing interface which defines interactions between multiple software intermediaries 
	\item \textbf{UI}: User interface	
	\item \textbf{GUI}: Graphical user interface
	\item \textbf{DB}: Database
	\item \textbf{REST}: Representational state transfer - software architectural style used in web services
\end{itemize}
\subsubsection{Abbreviations}
\begin{itemize}
	\item \textbf{App}: Application.
\end{itemize}

\newpage