\subsection{Overview}
\hspace{\parindent}Architectural design of the application is based on the three-layer model used in most applications. The three layers are the presentation layer, or frontend, the application layer, or middleware, and the data layer, or backend. Each of those layers does its own part of the job and communicates with other layers in order to present the correct information to the user.\\
The architecture is also a typical client-server implementation where server holds the data and the client is accessing it through requests (besides the offline mode where the user stored some of the data from the server locally and is accessing it without using online requests).\\
\subsubsection{Backend architecture}
\textbf{User database}\\

As previously mentioned, user database is located in the Google Firestore service. This service uses real-time NoSQL database. This database is organized in collection->document system. Everything starts with one collection, which can hold as many documents (entities) as possible. Each document can have an unlimited number of attribute fields, which can be with a specified type or without one, and at most one collection. Then the cycle repeats again.\\ \\
Working with NoSQL has its advantages when it doesn't have a lot of relational and connected data. We can extract only the small amount of data we want, it is very fast, and also very efficient. Since accounts are not connected in any way, using NoSQL database  seemed like a viable choice since it saves our users time and data. \\ \\
The architecture of this database is the following: the first collection contains all the users as their unique IDs, where they are represented by a document. The first level of the document holds only the info of the display name and the ID.  The second collection is made for storing user preferences, and every user has their own. In that collection there is a document that holds all of the parameters needed for the user's usage of the app, such as colour mode, economy level, thumbnail URL; and two optional ones, real name and real surname
\\ \\
The following figure represents the structure of the explained database.\\
IMAGE OF THE NoSQL architecture


\subsubsection{Middleware architecture}
In order to connect the data on the server to the screen of the phone and to allow the user to properly see the data, we have implemented a complicated layer of functions and classes in order to create easy-to-use and esthetically pleasing experience for the user.

